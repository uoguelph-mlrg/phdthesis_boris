% **************************** Custom Packages ********************************
\let\newfloat\undefined
\usepackage{algorithm}
\usepackage{algpseudocode}
\usepackage{amsfonts}       % blackboard math symbols
\usepackage{amsmath}
\usepackage{amssymb}
\usepackage{bbm}  % for \mathbbm{1}
\usepackage{bibentry}
\nobibliography*
%\newcommand\bmmax{2}
\newcommand\hmmax{0}
\newcommand\bmmax{0}
\usepackage{bm}  % for more powerful bold symbols https://tex.stackexchange.com/a/596
\usepackage{booktabs}
\usepackage{breakcites}
\usepackage{pifont}  % for dingbats checkmark and xmark
\usepackage{mathtools}
\usepackage{ifthen}
\usepackage{keyval}
\usepackage{placeins}
\usepackage{tabu}
\usepackage{tikz}
\usepackage[normalem]{ulem}
\usepackage{upgreek}
\usepackage{natbib}
\usepackage{nicefrac}       % compact symbols for 1/2, etc.
\usepackage{soul}
\usepackage{enumitem}
\usepackage{microtype}      % microtypography
\usepackage{xcolor,colortbl}
\usepackage{tikz}
%\usepackage{subfigure}
\usepackage{tabularx}
\usepackage{makecell}
\usepackage{titlesec}

%for background
%\DeclareMathOperator{\E}{\mathbb{E}}

\usepackage{wrapfig,lipsum,booktabs}

\usepackage{multirow}

\DeclareMathOperator*{\argmax}{argmax}
\DeclareMathOperator{\Tr}{Tr}  % GWT: Trace
\newcommand{\cmark}{\ding{51}}%
\newcommand{\xmark}{\ding{55}}%

\usepackage{capt-of}
\usepackage{times}
\usepackage{epsfig}
\usepackage[export]{adjustbox}  % for adjusting images
\usepackage{xspace}
\usepackage{setspace}
\usepackage{epigraph} 

%\newcommand*{\eg}{e.g.\@\xspace}
%\newcommand*{\ie}{i.e.\@\xspace}
%\newcommand*{\wrt}{w.r.t.\xspace}

% If you comment hyperref and then uncomment it, you should delete
% egpaper.aux before re-running latex.  (Or just hit 'q' on the first latex
% run, let it finish, and you should be clear).
\usepackage[pagebackref=true,breaklinks=true,colorlinks,bookmarks=false]{hyperref}

\usetikzlibrary{arrows.meta, backgrounds, calc, fit, positioning}

\DeclareMathOperator*{\binop}{\oplus}


% *************************** Graphics and figures *****************************

% Uncomment the following two lines to force Latex to place the figure.
% Use [H] when including graphics. Note 'H' instead of 'h'
%\usepackage{float}
%\restylefloat{figure}

\usepackage{caption}
\usepackage{subcaption}

% ********************************** Tables ************************************
\usepackage{booktabs} % For professional looking tables
% \usepackage{multirow}

%\usepackage{multicol}
%\usepackage{longtable}
%\usepackage{tabularx}

\usepackage{graphbox,graphicx}

% *********************************** SI Units *********************************
\usepackage{siunitx} % use this package module for SI units


% ******************************* Line Spacing *********************************

% Choose linespacing as appropriate. Default is one-half line spacing as per the
% University guidelines

% \DoubleSpacing
\setuogspacing{\OnehalfSpacing}
% \SingleSpacing


% ************************ Formatting / Footnote *******************************

% Don't break enumeration (etc.) across pages in an ugly manner (default 10000)
%\clubpenalty=500
%\widowpenalty=500

%\usepackage[perpage]{footmisc} %Range of footnote options


% *****************************************************************************
% *************************** Bibliography  and References ********************

%\usepackage{cleveref} %Referencing without need to explicitly state fig /table

% Add `custombib' in the document class option to use this section

% changes the default name `Bibliography` -> `References'
\renewcommand{\bibname}{References}


% ******************************** Roman Pages *********************************
% The romanpages environment set the page numbering to lowercase roman one
% for the contents and figures lists. It also resets
% page-numbering for the remainder of the dissertation (arabic, starting at 1).

\newenvironment{romanpages}{
  \setcounter{page}{1}
  \renewcommand{\thepage}{\roman{page}}}
{\newpage\renewcommand{\thepage}{\arabic{page}}}


% ******************************************************************************
% ************************* User Defined Commands ******************************
% ******************************************************************************

\makeatletter                  % You do not need to write [htpb] all the time
\renewcommand\fps@figure{htbp} %
\renewcommand\fps@table{htbp}  %
\makeatother                   %

% *********** To change the name of Table of Contents / LOF and LOT ************

%\renewcommand{\contentsname}{My Table of Contents}
%\renewcommand{\listfigurename}{My List of Figures}
%\renewcommand{\listtablename}{My List of Tables}
\DeclarePairedDelimiter\ceil{\lceil}{\rceil}
\DeclarePairedDelimiter\floor{\lfloor}{\rfloor}
\newcommand{\matr}[1]{#1}
\newcommand{\tens}[1]{\mathcal{#1}}
\renewcommand{\vec}[1]{#1}
\newcommand{\vect}[2]{\vec{#1}^{(#2)}}

% Generalized hadamard product fusion
\newcommand{\thetavec}{\ensuremath{\theta}}
\newcommand{\fusop}{\ensuremath{\mathcal{F_{\thetavec}}}}
\newcommand{\mutanfeat}[1]{\ensuremath{\tilde{#1}}}
\newcommand{\m}{\ensuremath{m}}
\newcommand{\n}{\ensuremath{n}}
\newcommand{\q}{\ensuremath{q}}
\renewcommand{\v}{\ensuremath{v}}
\newcommand{\z}{\ensuremath{z}}
%\newcommand{\A}{\ensuremath{A}}
\newcommand{\B}{\ensuremath{B}}
%\renewcommand{\C}{\ensuremath{C}}
\newcommand{\R}{\ensuremath{\mathbb{R}}}
\newcommand{\T}{\ensuremath{\mathcal{T}}}
\newcommand{\Tfibre}{\ensuremath{\T_{i_1 \cdots i_{n - 1} \,:\, i_{n + 1} \cdots i_N}}}
\newcommand{\TsizeIn}[1]{\ensuremath{I_1 \times \cdots \times I_{n - 1} \times #1 \times I_{n + 1} \times \cdots \times I_N}}
\newcommand{\binopb}{\ensuremath{\sideset{}{_b}\binop}}
\newcommand{\binopseq}{\ensuremath{{(\binopb)}_{b = 1}^B}}
\newcommand{\binoppartition}{\ensuremath{{\{\tuckbranch\}}_b}}
\newcommand{\binoppartB}{\ensuremath{\mathbb{B}}}
\newcommand{\binoppartBseq}{\ensuremath{{(\binoppartB_b)}_{b = 1}^B}}
\newcommand{\tuckbranch}{\ensuremath{\T_r^{\q{}\v{}}}}
\newcommand{\hadamardqv}{\ensuremath{N_r\mutanfeat{q} \odot M_r\mutanfeat{v}}}

% SST
%\newcommand{\real}{\mathbb{R}}
\newcommand{\yvosvalG}{68.1}
\newcommand{\davisvalG}{53.2}
\newcommand{\inputvar}{\mathcal{X}}
\newcommand{\outputvar}{\mathcal{Y}}
\newcommand{\vidfeat}{\mathcal{T}}
\newcommand{\objfeat}{\mathcal{R}}
\newcommand{\pixel}{p}
\newcommand{\querytensor}{\mathcal{Q}}
\newcommand{\keytensor}{\mathcal{K}}
\newcommand{\embtensor}{\mathcal{E}}
\newcommand{\valuetensor}{\mathcal{V}}
\newcommand{\objmean}{\mu_o}
\newcommand{\objcov}{\Sigma_o}
\newcommand{\objctx}{\tilde{\mu}_o}
\newcommand{\viddim}{\real{}^{C\times T\times H\times W}}
\newcommand{\softmax}{\mathtt{softmax}}
\newcommand{\yesmark}{\ding{51}}
\newcommand{\nomark}{\ding{55}}
\newcommand{\J}{$\mathcal{J}$}
\newcommand{\F}{$\mathcal{F}$}
%\renewcommand{\G}{\mathcal{G}}
\newcommand{\JandF}{$\mathcal{J} \& \mathcal{F}$}



\newcommand{\A}{\mathbf{A}}
\newcommand{\Lapl}{\mathbf{L}}
\newcommand{\D}{\mathbf{D}}
\newcommand{\G}{\mathcal{G}}
\newcommand{\V}{\mathcal{V}}
\newcommand{\E}{\mathcal{E}}
\newcommand{\Y}{\mathbf{Y}}
\newcommand{\y}{\mathbf{y}}
\newcommand{\X}{\mathbf{X}}
\newcommand{\x}{\mathbf{x}}
\newcommand{\Z}{\mathbf{Z}}
\newcommand{\h}{\mathbf{h}}
% \newcommand{\H}{\mathbf{H}}
\newcommand{\W}{\mathbf{W}}
\newcommand{\w}{\mathbf{w}}
%\newcommand{\R}{\mathbb{R}}
\newcommand{\eg}{\emph{e.g.}\@\xspace}
\newcommand{\ie}{\emph{i.e.}\@\xspace}
\newcommand{\wrt}{\emph{w.r.t.}\@\xspace}
\newcommand{\IID}{\emph{i.i.d.}\@\xspace}
\newcommand{\std}[1]{{\tiny{$\pm$#1}}}

% \newcommand{\cmark}{\ding{51}}%
% \newcommand{\xmark}{\ding{55}}%

\newcommand{\cnns}{CNNs\xspace}
\newcommand{\cnn}{CNN\xspace}
\newcommand{\gnns}{GNNs\xspace} % $\mathcal{G}$\textit{NNs}
\newcommand{\gnn}{GNN\xspace} % $\mathcal{G}$\textit{NN}

\newcommand{\cgshort}{\textsc{CoGen}\xspace}
\newcommand{\cg}{compositional generalization\xspace}
\newcommand{\CG}{Compositional generalization\xspace}

% NeurIPS 2019
\newcommand{\mnistfull}{\textsc{MNIST}\xspace}
\newcommand{\mnist}{\textsc{MNIST-75sp}\xspace}
\newcommand{\colors}{\textsc{Colors}\xspace}
\newcommand{\tri}{\textsc{Triangles}\xspace}
\newcommand{\collab}{\textsc{Collab}\xspace}
\newcommand{\proteins}{\textsc{Proteins}\xspace}
\newcommand{\dd}{\textsc{D\&D}\xspace}
\newcommand{\synthetic}{\colors, \tri~and \mnist\xspace}
\newcommand{\real}{\collab, \proteins~and \dd\xspace}
\newcommand{\wsup}{weakly-supervised\xspace}

% ICCV 2021
\newcommand{\graph}{\mathit{\cal G}}
\newcommand{\pgraph}{\mathit{\hat{\graph}}}
\newcommand{\structn}{\textsc{GraphN}\xspace}
\newcommand{\oracle}{\textsc{Oracle-Zs}\xspace}

\newcommand{\tmin}{t_\mathrm{min}}
\newcommand{\abs}[1]{\left\lvert#1\right\rvert}
\newcommand{\maxi}[1]{\overline{\mathbf{#1}}}
\newcommand{\mini}[1]{\underline{\mathbf{#1}}}
\newcommand{\interval}[2]{\left[\;#1\:,#2\;\right]}
\newcommand{\sumall}{\sum_{i=0}^s}
\newcommand{\sumj}{\sum_{\substack{i = 0\\i \neq j}}^s}
\newcommand{\di}{\sigma^i(\maxi{\boldsymbol\lambda})}
\newcommand{\djj}{\sigma^j(\maxi{\boldsymbol\lambda})}


% NeurIPS 2021
\newcommand{\mad}[1]{{\scriptsize{$\pm$#1}}}
\newcommand{\sem}[1]{{\scriptsize{$\pm$#1}}}
\newcommand{\topacc}[1]{{\scriptsize{/#1}}}
\newcommand{\f}{a}

\newcommand{\nets}{\mathcal{F}}
\newcommand{\domain}{\mathcal{D}}
\newcommand{\neigh}{\mathcal{N}}
%\newcommand{\h}{\mathbf{h}}
\newcommand{\loss}{{\cal L}}
%\newcommand{\w}{\mathbf{w}}

\newcommand{\params}{parameters\xspace}
\newcommand{\net}{architecture\xspace}

\newcommand{\PLH}{{\mkern-0.1mu\times\mkern-0.1mu}}
\newcommand{\ghnbase}{\textsc{GHN-1}\xspace}
\newcommand{\ghnours}{\textsc{GHN-2}\xspace}
\newcommand{\dataset}{\textsc{DeepNets-1M}\xspace}


\newcommand{\iid}{\textsc{ID}\xspace}
\newcommand{\iidtrain}{\textsc{Train}\xspace}
\newcommand{\iidval}{\textsc{Val}\xspace}
\newcommand{\iidtest}{\textsc{Test}\xspace}
\newcommand{\ood}{\textsc{OOD}\xspace}
\newcommand{\wide}{\textsc{Wide}\xspace}
\newcommand{\deep}{\textsc{Deep}\xspace}
\newcommand{\dense}{\textsc{Dense}\xspace}
\newcommand{\bnfree}{\textsc{BN-free}\xspace}

\newcommand{\bp}{\mathbf{p}}
\newcommand{\bt}{\mathbf{t}}
\newcommand{\bv}{\mathbf{v}}
\newcommand{\bw}{\mathbf{w}}
\newcommand{\bx}{\mathbf{x}}
\newcommand{\by}{\mathbf{y}}
\newcommand{\bz}{\mathbf{z}}

%\newcommand{\bm}{\mathbf{m}}
\newcommand{\bn}{\mathbf{n}}
\newcommand{\be}{\mathbf{e}}
\newcommand{\bc}{\mathbf{c}}

% Other stuff
\newcommand{\best}[1]{{\bfseries#1}}
\newcommand{\densepar}[1]{\textbf{#1}}
\newcommand{\apdx}{\textit{Appendix}} 

\definecolor{bad}{gray}{0.85}
%\definecolor{bad}{gray}{0.95}
\definecolor{extreme}{rgb}{0.82,0.82,0.90}
%\definecolor{bad}{rgb}{0.9,0.9,0.9}
\definecolor{good}{rgb}{0.85, 1.0, 0.85}
\newcommand\crule[3][black]{\textcolor{#1}{\rule{#2}{#3}}}

\newcommand\Tstrut{\rule{0pt}{3ex}}         % = `top' strut
\newcommand\Bstrut{\rule[-1.3ex]{0pt}{0pt}}   % = `bottom'

\newcommand\blfootnote[1]{%
  \begingroup
  \renewcommand\thefootnote{}\footnote{#1}%
  \addtocounter{footnote}{-1}%
  \endgroup
}

\newcommand{\venue}[1]{\ul{#1}}
\newcommand{\fig}[1]{Fig.~{#1}}
\newcommand{\secref}[1]{\S~{#1}}

\setsecnumdepth{subsection}

\renewcommand{\toprule}{\Xhline{3\arrayrulewidth} }
\renewcommand{\midrule}{\Xhline{2\arrayrulewidth} }
\renewcommand{\bottomrule}{\Xhline{3\arrayrulewidth} }

\usepackage{stfloats}
\fnbelowfloat % puts footnotes below the bottom floats
\makeatletter
\def\blfootnote{\xdef\@thefnmark{}\@footnotetext}
\makeatother

%\newcommand{\eqref}[]{}

% ********************************* Appendix ***********************************

% The default value of both \appendixtocname and \appendixpagename is `Appendices'. These names can all be changed via:

%\renewcommand{\appendixtocname}{List of appendices}
%\renewcommand{\appendixname}{Appndx}
